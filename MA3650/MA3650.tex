\documentclass{article}
\usepackage[utf8]{inputenc}
\usepackage{url,amsmath,graphicx,amssymb,booktabs}
\usepackage[top=1.5cm, bottom=1.5cm, left=2.5cm, right=2.5cm]{geometry}

\title{MA3650 - Numerical Methods for Differential Equations}
\author{Luke Dando}
\date{\today}

\begin{document}

\maketitle

\tableofcontents

\section{Revision of Taylor series and Newton's method}
\subsection{Lecture 1}


\subsection{Lecture 2}


\section{Convergence of Newton's method}
\subsection{Lecture 3}
\subsection{Lecture 4}


\section{Wrapping up chapter 1}
\subsection{Lecture 6}
\subsection{Summary}


\section{Finite differences}
\subsection{Lecture 7}
\subsection{Lecture 8}


\section{Explicit numerical schemes for the heat equation}
\subsection{Lecture 9}
\subsection{Lecture 10}


\section{Further numerical schemes for the heat equation}
\subsection{Lecture 11}
\subsection{Lecture 12}


\section{Numerical schemes for complex boundary conditions}
\subsection{Lecture 13}


\section{Numerical schemes for complex boundary conditions 2}
\subsection{Lecture 14}
\subsection{Lecture 15}
\subsection{Lecture 16}


\section{Matrix norms}
\subsection{Lecture 17}


\section{Matrix perturbations and condition numbers}
\subsection{Lecture 18}
\subsection{Lecture 19}


\section{Interpolation, Lagrange polynomials}
\subsection{Lecture 21}
\subsection{Lecture 22}


\section{Interpolation, splines}
\subsection{Lecture 23}
\subsection{Lecture 24}


\section{Different schemes for the heat equation, consistency}
\subsection{Lecture 25}


\section{Consistency}
\subsection{Lecture 26}


\section{Convergence}
\subsection{Lecture 27}
\subsection{Lecture 28}


\section{Stability Fourier method}
\subsection{Lecture 29}
\subsection{Lecture 30}
\subsection{Lecture 31}


\section{Stability matrix method}
\subsection{Lecture 32}
\subsection{Lecture 33}
\subsection{Lecture 34}


\section{Polynomial Interpolation}
\subsection{Lagrange Polynomial}
For the Lagrange polynomials, $y_i$ and $f(x_i)$ can be used interchangeably.
Given $n+1$ points $(x_i,y_i) \in \mathbb{R}^2,\,0\leq i\leq n$, with $x_i\neq x_j$ when $i\neq j$. The polynomials
\begin{equation}
    L_i(x) = \prod_{j=0,\,\neq i}^n \left(\frac{x-x_j}{x_i-x_j}\right),\quad 0\leq i \leq n,
\end{equation}
satisfy $L_i(x_i)=1$ and $L_i(x_j)=0$ for $j\neq i$.
Now, given $n+1$ points $(x_i,y_i) \in \mathbb{R}^2,\,0\leq i\leq n$, with $x_i\neq x_j$ when $i\neq j$. The Lagrange polynomial $p$ is a polynomial of degree up to $n$ equal to
\begin{equation}
    p(x) = \sum_{i=0}^n y_iL_i(x).
\end{equation}
This is \textit{linear} if $n=1$ and \textit{quadratic} if $n=2$.
\subsection{Vandermonde Method}
Given $p(x)=ax^2+bx+c$, with $a,\,b$ and $c$ to be determined from the conditions $p(x_i)=y_i,\, i=0,\,1,\,2$, we can input this into a matrix know as the Vandermonde matrix and solve it for values $a,\,b$ and $c$.
\begin{equation}
    \begin{pmatrix}  x_0^2 & x_0 & 1 \\ x_1^2 & x_1 & 1 \\ x_2^2 & x_2 & 1 \end{pmatrix}\begin{pmatrix} a \\ b \\ c \end{pmatrix} = \begin{pmatrix} y_0 \\ y_1 \\ y_2 \end{pmatrix}.
\end{equation}
This can be extended for larger polynomials trivially.

\end{document}
